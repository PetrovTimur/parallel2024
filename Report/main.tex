\documentclass[12pt, a4paper]{article}

\usepackage[utf8]{inputenc}
\usepackage[russian]{babel}

\usepackage{graphicx}
\usepackage{amsfonts}
\usepackage{amsmath}
\usepackage{cmap}
\usepackage[hidelinks]{hyperref}
\usepackage{epstopdf}
\usepackage{cite}
\usepackage{indentfirst}
\usepackage{listings}
\usepackage{xcolor}

\usepackage[left=2cm,right=1cm,
top=2cm,bottom=2cm,bindingoffset=0cm]{geometry}

\graphicspath{{./images/}{./course}}

\definecolor{codegreen}{rgb}{0,0.6,0}
\definecolor{codegray}{rgb}{0.5,0.5,0.5}
\definecolor{codepurple}{rgb}{0.58,0,0.82}
\definecolor{backcolour}{rgb}{0.95,0.95,0.92}

\lstdefinestyle{mystyle}{
	backgroundcolor=\color{backcolour},   
	commentstyle=\color{codegreen},
	keywordstyle=\color{magenta},
	numberstyle=\tiny\color{codegray},
	stringstyle=\color{codepurple},
	basicstyle=\ttfamily\footnotesize,
	breakatwhitespace=false,         
	breaklines=true,                 
	captionpos=b,                    
	keepspaces=true,                                  
	showspaces=false,                
	showstringspaces=false,
	showtabs=false,                  
	tabsize=2
}

\begin{document}
	
\thispagestyle{empty}

\begin{center}
	\ \vspace{-3cm}
	
	\includegraphics[width=0.5\textwidth]{msu.eps}\\
	{Московский государственный университет имени М. В. Ломоносова}\\
	Факультет вычислительной математики и кибернетики\\
	Кафедра вычислительных методов
	
	\vspace{6cm}
	
	{\Large \bfseries Построение разреженной матрицы и решение СЛАУ}
	
	\vspace{1cm}
	
	{\large Параллельные высокопроизводительные вычисления}
\end{center}

\vfill

\begin{flushright}
	\textbf{выполнил:}\\
	Петров Тимур \\
	группа 504
\end{flushright}

\vfill

\begin{center}
	31 октября \\
	Москва, 2024
\end{center}

\enlargethispage{2\baselineskip}

\newpage

\tableofcontents

\newpage

\section{Описание задания и программной реализации}

\subsection{Краткое описание задания}

\subsection{Краткое описание программной реализации}

Для запуска на локальных системах используется следующая команда:

\begin{lstlisting}{bash}
	OMP_NUM_THREADS=k ./a.out Nx Ny K1 K2
\end{lstlisting}

Для запуска на кластере, использующем систему очередей, запускается скрипт со следующими параметрами.

\subsection{Описание опробованных способов оптимизации последовательных вычислений}

sdfs

\newpage

\section{Исследование производительности}

\subsection{Характеристики вычислительной системы}

\begin{center}
	\begin{tabular}{ c|c|c|c| } 
		 & ноутбук & десктоп & кластер \\ 
		\hline
		CPU & i7-7700H & i5-12400F & IBM POWER 8 \\ 
		Cores & 4 & 6 & 20 \\ 
		Threads & 2 & 2 & 8 \\
		Peak perf &  &  &  \\
		RAM &  & DDR5-5600 2x16 &  \\
		transfer rate &  &  &  \\
		OS & 22.04 & 22.04  &  \\
	\end{tabular}
\end{center}

\lstset{style=mystyle}

\subsection{Результаты измерений производительности}

sdfs

\newpage

\section{Анализ полученных результатов}

adssad

\end{document}